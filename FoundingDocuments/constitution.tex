\documentclass[11pt]{article}


\usepackage{enumerate}


\title{\textbf{VictorCrypto Constitution: DRAFT}}

\newcommand{\orgname}{\textbf{VictorCrypto}}

\begin{document}
    \maketitle


    \section{Preamble}
    We, the members of \orgname, do hereby establish this constitution.


    \section{Article I: Name}

    The name of the organization will be \orgname.


    \section{Article II: Affiliation with other groups}

    \orgname has no affiliation with the University of Michigan or other organizations at this time.


    \section{Article III: Mission/Purpose Statement}

    Section 1. Purpose: \orgname is established for the purpose of encouraging students to be involved with their
    campus community and to promote an environment of learning.

    Section 2. Mission: \orgname 's mission is to provide a space to discuss, research, and study cryptographical and privacy topics.

    Section 3. \orgname understands and is committed to fulfilling its responsibilities of abiding by the University
    of Michigan policies and procedures.



    \section{Article IV: Membership}

    \subsection{Active membership shall include:}

    Active membership shall be limited to persons officially connected with the University of Michigan as faculty, staff, or registered students.
    In addition, the following requirements are necessary to constitute active members:

    \begin{enumerate}
        [a)]
        \item Attendance of at least 50\% of all campus events during a given semester
        \item Active participation in at least one sponsored activity or event
    \end{enumerate}

    Those non associated with the University of Michigan may join and participate in events but will not be considered "members".

    \subsection{Joining the organization:}

    Interested individuals may join the organization by attending a meeting, contacting a member, or contacting the
    group.
    To be considered active, the individual must meet the requirements outlined in Article IV, Section 1.

    \subsection{Member Withdrawal}

    A member may voluntarily withdraw from the organization by notifying the executive board of their desire to discontinue membership.

    \subsection{Non-discrimination and Non-disruption. }

    \orgname is committed to a policy of equal opportunity for all persons and does not discriminate on the basis of race, color, national origin, age, marital status, sex, sexual orientation, gender identity, gender expression, disability, religion, height, weight, or veteran status in its membership or activities.
    Upon joining the organization, all members agree not to undermine the purpose or mission of \orgname.


    \section{Article V: Officers}

    \orgname will be governed by an executive committee consisting of officers:

    \begin{enumerate}
        [a)]
        \item An elected President will preside at all meetings.
        \item \orgname shall also elect a Vice President.
        The Vice President's duties shall be to preside at all meetings and functions that the President cannot attend.
        \item \orgname shall also elect a Secretary-Treasurer who will handle all dues, accounts, new members, rule observances at state meetings, protocol, compliance with university policies, etc.
    \end{enumerate}

    Additionally, the following officer positions may or may not be filled at any given time

    \begin{enumerate}
        [a)]
        \item Head of Primitive Cryptography - Manages all things related to highly mathematical primitive cryptography.
        This includes finding and reading papers related to cryptographical primitives.
        \item Head of Applied Cryptography - Manages all things related to high level protocols cryptographical primitives.
        This includes finding and reading papers related to applied cryptography.
        \item Head of Policy, Privacy, and Ethics -  Manages all things related to policy, privacy, and ethics discussions.
        This includes finding and reading bills and laws related to privacy, encryption, export laws and more.
        This officer is also responsible for promoting a culture of diversity, equity, and inclusion.
        Any moral or ethical objections to club activities made by this officer must be made directly to the
        President, Vice President, and all Advisors and a meeting must be scheduled with all officers to discuss and
        resolve the matter.
    \end{enumerate}


    \section{Article VI: Advisor}

    The role and duties of the faculty/staff advisor shall include attending meetings, providing counsel to the executive committee, and advising the organization on policies, decision making, and leadership development.
    A willing faculty or staff member can be asked to serve in the role by the executive board in consultation with the general body members.
    The advisor shall not vote on official organization business.


    \section{Article VII: Operations}

    \begin{enumerate}[I.]
        \item Voting Eligibility:

        Those members meeting all requirements of active membership as set forth in Article IV Section 1 and have
        attended at least 2 meetings will be granted voting privileges.

        \item Election process:
        \begin{enumerate}
            [a)]
            \item All officers shall be elected by a majority vote of the eligible voting members of \orgname.
            All elections will be held on an annual basis during the month of August.
            Incoming officers will assume their positions one month after elections are held.
            \item The outgoing President will take nominations from the floor.
            The nomination process must be closed and the movement seconded.
            The nominated parties will be allowed to vote.
            \item All voting shall be done by secret ballot to be collected and tabulated by the outgoing
            Secretary-Treasurer and one other voting member of \orgname appointed by the outgoing President.
            If the outgoing Secretary-Treasurer is a nominee for another position, they should recuse themselves of tallying ballots and appoint another officer to take their place.
        \end{enumerate}
        \item Meetings

        All meetings will occur on a weekly basis at a time selected by the executive committee and will follow the procedure set forth below.

        \begin{enumerate}
            [i)]
            \item Attendance
            \item Report by the President
            \item Committee Reports
            \item Vote on all committee motions/decisions
            \item Any other business
            \item Dismissal by the President
        \end{enumerate}

    \end{enumerate}


    \section{Article VIII: Removal of Membership or Officers.}

    \begin{enumerate}[I.]
        \item Removal of Officers: Any officer of \orgname in violation of \orgname's purpose, constitution, or who
        fails to fulfill their responsibilities as outlined in Article V, may be removed from office by the following process:
        \begin{enumerate}[a.]
            \item A written request by at least three members of the organization.
            \item Written notification to the officer of the request, asking the officer to be present at the next meeting and prepared to speak.
            \item A two-thirds (2/3) majority vote of eligible voting members is necessary to remove the officer.
        \end{enumerate}
        \item II. Removal of Membership: Any member of \orgname in violation of \orgname's purpose, constitution, or
        who fails to meet the membership requirements as outlined in Article IV, may have their privileges as a member revoked through the following process:
        \begin{enumerate}[a.]
            \item A written request by at least three members of the organization must be sent to the executive
            committee.
            \item Written notification to the member in question, asking the member to be present the next executive
            committee meeting and be prepared to speak.
            \item A unanimous decision by the executive committee members, in consultation with the advisor, to remove
            the member from the organization.
        \end{enumerate}
    \end{enumerate}


    \section{Article X: Amendments and Ratification}
    \begin{enumerate}[I.]
        \item This constitution is binding to all members of \orgname but it is not binding unto itself.
        Amendments to the constitution may be proposed in writing by any voting member of \orgname at any time via a pull request to this repository.
        The amendments will be placed on the agenda for the next regular meeting and voted on at that time.
        \item Proposed amendments will become immediately effective following approval of two-thirds (2/3) vote of all active members.
        \item This constitution must be ratified by two-thirds of all active members to take effect and shall be reviewed every 3 years.
    \end{enumerate}


    \section{Article XI: Statement of Compliance.}
    \orgname has read and agrees to fully comply with the University's policies.
    We understand that the organization's registration is contingent upon acceptance of these policies.
\end{document}